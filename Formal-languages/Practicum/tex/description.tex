\documentclass[10pt,a4paper]{article}
\usepackage[utf8]{inputenc}
\usepackage[russian]{babel}
\usepackage[OT1]{fontenc}
\usepackage{amsmath}
\usepackage{amsfonts}
\usepackage{amssymb}
\author{Романенко Илья}
\title{ Практикум }
\date{}
\begin{document}
\maketitle

Построим автомат по исходному регулярному выражению. Тогда условие, что в регулярном языке есть слово с суффиксом $x^k$ эквивалентно тому, что в языке суффиксов исходного регулярного языка есть слово $x^k$. Таким образом алгоритм состоит в следующем - по исходной польской нотации построить автомат, потом провести $\varepsilon$ переходы из начального состояния во все состояния автомата (это как раз эквивалентно принятию автоматом всех суффиксов исходного языка). Затем проверить, что слово $x^k$ действительно принадлежит получившемуся автомату.

\end{document}